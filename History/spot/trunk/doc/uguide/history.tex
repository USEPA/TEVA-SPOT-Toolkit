TODO Update Release History

\section{Release Notes}\label{history_introNotes}
The section describing each section includes some notes and caveats for the use of the different solvers. These solvers have been tested on simple examples, as well as on some moderate-\/scale water networks (e.g. network1). Current efforts have focused on ensuring the correctness of these solvers, and their scalability to problems of this scale. The notes in this document identify additional limitations that we know about and which will be addressed in subsequent releases.

\subsection{Release 1.0}\label{history_spot1_0}

Rename of TEVA-\/SP to SPOT

Added reduced-\/memory mode for heuristic solves.

Improved aggregation mechanism for integer programming solves. Added a new aggregation threshold that supports the computation of bounds on the final solution in a natural manner.

Integrated PICO 1.1, which works on PC platforms.

Added timing control for SPOT solvers, and instrumented all SPOT solvers to write out the best solution seen so far as soon as it is found.

Integrated bound computation for aggregated integer programming solves. Instrumented the PICO solver to report when a valid lower bound has been computed.

Added the 'awd' performance goal for 'sp', which can be used for tracer-\/study sensor placement optimization. Created the sp-\/tracer script to apply this functionality in a simple manner.

\subsection{Release 2.0}\label{history_tevasp2_0}

\begin{verbatim}

   Added heuristic support to optimize sensor placement with imperfect sensors

   Improved heuristic support for VaR, TCE, and Worst-Case performance

   Improved IP solver support for witness aggregation

   See examples 2-5 in the "sp executable" documentation 
\end{verbatim}

\subsection{Release 1.1}\label{history_tevasp1_1_0}

\begin{verbatim}
TODO:
  JP's notes:
    Added heuristic support for the VaR, TCE, and Worst-Case performance 
    statistics. 

    Added heuristic support for invalid/fixed facility locations.

  Bill's notes:
    Added --seed option to 'sp' to enable control of pseudo-random events
	in the sensor placement heuristic.

    Removed the --fixed* and --invalid* options to 'sp'.  These have been
	replaced with the --sensor-locations option, which specifies
	a file that can specify whether locations are feasible/infeasible
        and fixed/unfixed.

    Added the --gamma option to 'sp', which specifies the percent of the
	tail distribution used for robustness computations: var, cvar, tce.

    Added the --numsamples option to 'sp', which controls the number of
	iterations used in the heuristic sensor placement solver.
    
\end{verbatim}

\subsection{Release 1.0.2}\label{history_tevasp1_0_2}

\begin{verbatim}
November 30, 2005

Changes in this release:

Bug fixes:

  - Fixed the appearance on some test platforms of nan values in
    the impact file for the population exposed objective on a test 
    network and threat.

Performance improvements:

  - Large improvement for generation of impact files
    (with tso2Impact) for the population exposed objective.  
    Somewhat smaller improvement for population killed objective.

Enhancements:

  - Changes to enable use of CVaR measure within the integer 
    programming solver.

  - Rework of GRASP in preparation for incorporation of new objectives.

  - Added new tests (test1i, test1j and test1k) to sp/test.

Portability issues:

  - Some 64 bit portability changes to utilib

  - Minor fix to sp script so AMPL/PICO solver executes properly on 
    Windows

Known issues in this release:

  The PICO executable under Windows has a known issue when solving 
  for the mass consumed objective with the CVaR statistics.
  The executable released with 1.0.2 does not exhibit the problem, 
  at the cost of speed.  When the problem is solved, a faster PICO 
  executable will be available.
\end{verbatim}

\subsection{Release 1.0.1}\label{history_tevasp1_0_1}

\begin{verbatim}
October 31, 2005

Performance of evalsensor is improved.
\end{verbatim}

\subsection{Release 1.0.0}\label{history_tevasp1_0_0}

\begin{verbatim}
October 28, 2005

Initial release.
\end{verbatim}
 
